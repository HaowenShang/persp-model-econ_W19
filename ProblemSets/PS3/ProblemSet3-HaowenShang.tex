\documentclass[letterpaper,12pt]{article}
\usepackage{array}
\usepackage{threeparttable}
\usepackage{geometry}
\geometry{letterpaper,tmargin=1in,bmargin=1in,lmargin=1.25in,rmargin=1.25in}
\usepackage{fancyhdr}
\usepackage{lastpage}
\pagestyle{fancy}
\lhead{}
\chead{}
\rhead{}
\lfoot{}
\cfoot{}
\rfoot{\footnotesize\textsl{Page \thepage\ of \pageref{LastPage}}}
\renewcommand\headrulewidth{0pt}
\renewcommand\footrulewidth{0pt}
\usepackage[format=hang,font=normalsize,labelfont=bf]{caption}
\usepackage{listings}
\lstset{frame=single,
  language=Python,
  showstringspaces=false,
  columns=flexible,
  basicstyle={\small\ttfamily},
  numbers=none,
  breaklines=true,
  breakatwhitespace=true
  tabsize=3
}
\usepackage{amsmath}
\usepackage{amssymb}
\usepackage{amsthm}
\usepackage{harvard}
\usepackage{setspace}
\usepackage{float,color}
\usepackage[pdftex]{graphicx}
\usepackage{hyperref}
\hypersetup{colorlinks,linkcolor=red,urlcolor=blue}
\theoremstyle{definition}
\newtheorem{theorem}{Theorem}
\newtheorem{acknowledgement}[theorem]{Acknowledgement}
\newtheorem{algorithm}[theorem]{Algorithm}
\newtheorem{axiom}[theorem]{Axiom}
\newtheorem{case}[theorem]{Case}
\newtheorem{claim}[theorem]{Claim}
\newtheorem{conclusion}[theorem]{Conclusion}
\newtheorem{condition}[theorem]{Condition}
\newtheorem{conjecture}[theorem]{Conjecture}
\newtheorem{corollary}[theorem]{Corollary}
\newtheorem{criterion}[theorem]{Criterion}
\newtheorem{definition}[theorem]{Definition}
\newtheorem{derivation}{Derivation} % Number derivations on their own
\newtheorem{example}[theorem]{Example}
\newtheorem{exercise}[theorem]{Exercise}
\newtheorem{lemma}[theorem]{Lemma}
\newtheorem{notation}[theorem]{Notation}
\newtheorem{problem}[theorem]{Problem}
\newtheorem{proposition}{Proposition} % Number propositions on their own
\newtheorem{remark}[theorem]{Remark}
\newtheorem{solution}[theorem]{Solution}
\newtheorem{summary}[theorem]{Summary}
%\numberwithin{equation}{section}
\bibliographystyle{aer}
\newcommand\ve{\varepsilon}
\newcommand\boldline{\arrayrulewidth{1pt}\hline}


\begin{document}

\begin{flushleft}
  \textbf{\large{Problem Set \#3}} \\
  MACS 30150, Dr. Evans \\
  Haowen Shang
\end{flushleft}

\vspace{5mm}

\noindent\textbf{Exercise 5.1}

The condition that characterizes the optimal amount of cake to eat in period 1 is :

\begin{equation*}
  \max_{c_{1}\in \left [ 0,W_{1} \right ]}u\left ( c_{1} \right )\quad s.t. \quad W_{2} = W_{1} - c_{1}
\end{equation*}

The condition for the optimal amount of cake to save for the next period $W_{2}$ is:

\begin{equation*}
  \max_{W_{2}\in \left [ 0,W_{1} \right ]}u\left ( W_{1}-W_{2} \right )
\end{equation*}

In order to maximize utility, we know that If the individual lives for one period, the optimal decision is:  $c_{1} = W_{1}$ and $W_{2} = 0$.


\noindent\textbf{Exercise 5.2}

The condition for the optimal amount of cake to leave for the next period $W_{3}$ in period 2 is:

\begin{equation*}
  \max_{W_{3}\in \left [ 0,W_{2} \right ]}u\left ( W_{2}-W_{3} \right )
\end{equation*}

In order to maximize utility, we know that If the individual lives for two period, in period 2, the optimal decision is:  : $c_{2} = W_{2}$ and $W_{3} = 0$.

The condition for the optimal amount of cake leave for the next period $W_{2}$ in period 1 is:

\begin{equation*}
  \max_{W_{2}\in \left [ 0,W_{1} \right ]} \left [  u\left ( W_{1}-W_{2} \right )+\beta\cdot \max_{W_{3}\in \left [ 0,W_{2} \right ]} u\left ( W_{2}-W_{3}\right )\right ]
\end{equation*}


Since we know $W_{3} = 0$, the formular above is:

\begin{equation*}
  \max_{W_{2}\in \left [ 0,W_{1} \right ]} \left [  u\left ( W_{1}-W_{2} \right )+\beta\cdot  u\left ( W_{2} \right )\right ]
\end{equation*}

Then we get the first order condition of period 1:

\begin{equation*}
   u'\left ( W_{1}-W_{2} \right )=\beta\cdot  u'\left ( W_{2} \right )
\end{equation*}

If the utility function and $W_{1}$ are known, we can know what $W_{2}$ is, which means $W_{2} = \psi_1 (W_{1})$.

\noindent\textbf{Exercise 5.3}

The condition for the optimal amount of cake to leave for the next period $W_{4}$ in period 3 is:

\begin{equation*}
  \max_{W_{4}\in \left [ 0,W_{3} \right ]}u\left ( W_{3}-W_{4} \right )
\end{equation*}

In order to maximize utility, we know that if the individual lives for three period, in period 3, $c_{3} = W_{3}$ and $W_{4} = 0$

The condition for the optimal amount of cake leave for the next period $W_{3}$ in period 2 is:

\begin{equation*}
  \max_{W_{3}\in \left [ 0,W_{2} \right ]} \left [  u\left ( W_{2}-W_{3} \right )+\beta\cdot \max_{W_{4}\in \left [ 0,W_{3} \right ]} u\left ( W_{3}-W_{4} \right )\right ]
\end{equation*}

Since we know in last period, $W_{4} = 0$, the formular above is:

\begin{equation*}
  \max_{W_{3}\in \left [ 0,W_{2} \right ]} \left [  u\left ( W_{2}-W_{3} \right )+\beta\cdot  u\left ( W_{3} \right )\right ]
\end{equation*}

Then we get the first order condition of period 2:

\begin{equation*}
   u'\left ( W_{2}-W_{3} \right )=\beta\cdot  u'\left ( W_{3} \right )
\end{equation*}

Then we can get $W_{3} = \psi_2 (W_{2})$

The condition for the optimal amount of cake leave for the next period $W_{2}$ in period 1 is:

\begin{equation*}
  \max_{W_{2}\in \left [ 0,W_{1} \right ]} \left [  u\left ( W_{1}-W_{2} \right )+\beta\cdot\max_{W_{3}\in \left [ 0,W_{2} \right ]} u\left ( W_{2}-W_{3} \right )+\beta ^{2}\cdot \max_{W_{4}\in \left [ 0,W_{3} \right ]}u\left ( W_{3}-W_{4} \right )\right ]
\end{equation*}

Since we know $W_{4} = 0$ and $W_{3} = \psi_2 (W_{2})$, the formular above is:

\begin{equation*}
  \max_{W_{2}\in \left [ 0,W_{1} \right ]} \left [  u\left ( W_{1}-W_{2} \right )+\beta\cdot u\left ( W_{2}-\psi_2 (W_{2} ) \right )+\beta ^{2}\cdot u\left ( \psi_2 (W_{2})\right )\right ]
\end{equation*}

Then we get the first order condition for period 1:

\begin{equation*}
\begin{aligned}
   u'\left ( W_{1}-W_{2} \right ) = \beta\cdot u'\left ( W_{2}-\psi_2 (W_{2} ) \right )\cdot(1-\psi_{2}' (W_{2})) + \beta ^{2}\cdot u'(\psi_2 (W_{2}))\cdot \psi_{2} '(W_{2})\\= \beta\cdot u'\left ( W_{2}-W_{3}  \right )\cdot(1-\psi_{2}' (W_{2})) + \beta ^{2}\cdot u'(W_{3})\cdot \psi_{2} '(W_{2})\\=\beta\cdot u'\left ( W_{2}-W_{3}  \right ) +\beta\cdot \psi_{2}' (W_{2})\cdot [-u'( W_{2}-W_{3}) + \beta\cdot u'(W_{3})]\\=\beta\cdot u'\left ( W_{2}-W_{3}  \right ) 
 \end{aligned}
\end{equation*}

If the initial cake size is $W_{1} = 1$, the discount factor is $\beta = 0.9$, and the period utility function is $ln(c_{t})$, we can solve the equations above and get $W_{1} = 1$, $W_{2} =0.631 $, $W_{3} = 0.299$, and $W_{4} = 0$. And $c_{1} = 0.369$, $c_{2} =0.332$, $c_{3} = 0.299$.

The evolvement of $\left \{ c_{t} \right \}_{t=1}^{3}$ and $\left \{ W_{t} \right \}_{t=1}^{4}$ over the three periods shows in the JupyterNotebook.

\noindent\textbf{Exercise 5.4}

The condition that characterizes the optimal choice (the policy function) in period T-1 for $ W_T =\psi_{T-1} \left(W_{T-1}\right) $ is :

\begin{equation*}
  \max_{W_{T}\in \left [ 0,W_{T-1} \right ]}[u ( W_{T-1}-W_{T})+\beta \cdot u(W_T)]
\end{equation*}

Then we get the first order condition:

\begin{equation*}
u'(W_{T-1}-W_{T})=\beta\cdot  u'( W_{T} )
\end{equation*}

\begin{equation*}
u'(W_{T-1}-\psi_{T-1} (W_{T-1}))=\beta\cdot  u'( \psi_{T-1} (W_{T-1}) )
\end{equation*}

The value function $V_{T-1}$ is:

\begin{equation*}
V_{T-1}(W_{T-1}) = u(W_{T-1}-\psi_{T-1} (W_{T-1})) + \beta\cdot  u( \psi_{T-1} (W_{T-1}) )
\end{equation*}

\noindent\textbf{Exercise 5.5}

Since $u(c)=ln(c)$, 

in period T, we know that $$V_T(\overline{W}) = u(\overline{W})= ln(\overline{W})$$
And $$\psi_T (\overline{W}) = 0 $$

In period T-1, from the equation in Exercise 5.4, we can get $$\psi_{T-1} (\overline{W}) = \frac{\beta }{1+\beta }\cdot \overline{W}$$

And $$V_{T-1}(\overline{W}) = ln(\frac{1}{1+\beta }\overline{W})+\beta \cdot ln(\frac{\beta }{1+\beta }\overline{W})$$

Thus they are not equal.

\noindent\textbf{Exercise 5.6}
In period T-2,

\begin{equation*}
  \max_{W_{T-1}\in \left [ 0,W_{T-2} \right ]} \left [  u\left ( W_{T-2}-W_{T-1} \right )+\beta\cdot u\left ( W_{T-1}-\frac{\beta }{1+\beta }\cdot W_{T-1} \right )+\beta ^{2}\cdot u(\frac{\beta }{1+\beta }\cdot W_{T-1}  )\right ]
\end{equation*}


Then using the envelope theorem(like the equation in Exercise 5.3), we can get the first order condition:

\begin{equation*}
u'(W_{T-2}-W_{T-1})=\beta\cdot u'(W_{T-1}-\frac{\beta }{1+\beta }\cdot W_{T-1}) = \beta\cdot u'(\frac{1 }{1+\beta }\cdot W_{T-1})
\end{equation*}

Since $u(c)=ln(c)$, from the above equation we can get:

\begin{equation*}
  W_{T -1} = \psi_{T -2} (W_{T -2}) = \frac{\beta +\beta ^2}{1+\beta +\beta ^2}\cdot W_{T -2}
\end{equation*}

and

\begin{equation*}
V_{T-2}=ln( \frac{1}{1+\beta +\beta ^2}\cdot W_{T-2}) + \beta \cdot ln(\frac{\beta }{1+\beta +\beta ^2}\cdot W_{T -2})+ \beta^2 \cdot ln( \frac{\beta ^2}{1+\beta +\beta ^2}\cdot W_{T -2})
\end{equation*}

\noindent\textbf{Exercise 5.7}
For the general integer $s \geq 1$ using induction, we can get:

\begin{equation*}
\psi_{T-s}(W_{T-s})=\frac{\sum\limits_{i=1}^{s}\beta^{i}}{\sum\limits_{j=0}^{s}\beta^{j}}W_{T-s}
\end{equation*}

\begin{equation*}
V_{T-s}(W_{T-s}) = \sum\limits_{i=0}^{s} \beta^{i}ln\left(\frac{\beta^{i}\cdot W_{T-s}}{\sum\limits_{j=0}^{s}\beta^{j}}\right)
\end{equation*}

As s becomes infinite, 

\begin{equation*}
\psi_{T-s}(W_{T-s})= \psi(W_{T-s})=\frac{\frac{\beta }{1-\beta }}{\frac{1}{1-\beta }}\cdot W_{T-s}=\beta\cdot W_{T-s}
\end{equation*}

\begin{equation*}
\begin{aligned}
V_{T-s}(W_{T-s}) = V(W_{T-s}) =\sum\limits_{i=0}^{s} \beta^{i}ln(\beta^{i})+  \sum\limits_{i=0}^{s} \beta^{i}ln(\frac{1}{\sum\limits_{j=0}^{s}\beta^{j}})+\sum\limits_{i=0}^{s} \beta^{i} ln(W_{T-s}) \\
=\frac{\beta }{(1-\beta)^2}ln(\beta) +\frac{1 }{1-\beta}ln(1-\beta )+\frac{1 }{1-\beta}ln(W_{T-s})
\end{aligned}
\end{equation*}



As the horizon becomes further and further away (infinite), the value function and policy function become independent of time.


\noindent\textbf{Exercise 5.8}

$$V(W)\equiv \max_{W'\in [ 0,W]}u(W-W')+\beta V(W') $$
$W'$ is cake to leave for the next period.
\end{document}

